\documentclass[12pt,a4paper]{article}

% Paquetes esenciales
\usepackage[utf8]{inputenc}
\usepackage[spanish]{babel}
\usepackage[margin=2.5cm]{geometry}
\usepackage{graphicx}
\usepackage{hyperref}
\usepackage{listings}
\usepackage{xcolor}
\usepackage{booktabs}
\usepackage{array}
\usepackage{longtable}
\usepackage{fancyhdr}
\usepackage{titlesec}
\usepackage{enumitem}
\usepackage{parskip}
\usepackage{float}
\usepackage{amssymb}

% Configuración de colores (todo negro)
\definecolor{akemyred}{HTML}{000000}
\definecolor{codebg}{HTML}{F5F5F5}
\definecolor{codegreen}{HTML}{000000}
\definecolor{codepurple}{HTML}{000000}
\definecolor{codegray}{HTML}{555555}

% Configuración de hyperref
\hypersetup{
    colorlinks=true,
    linkcolor=black,
    filecolor=black,
    urlcolor=blue,
    citecolor=black,
    pdftitle={Manual Técnico AKEMY},
    pdfauthor={Junior Beltran Huaraya Chipana}
}

% Configuración de listings para código
\lstset{
    backgroundcolor=\color{codebg},
    basicstyle=\ttfamily\footnotesize,
    breaklines=true,
    captionpos=b,
    commentstyle=\color{codegray},
    keywordstyle=\bfseries,
    stringstyle=\color{black},
    numberstyle=\tiny\color{codegray},
    numbers=left,
    numbersep=5pt,
    frame=single,
    rulecolor=\color{black!30},
    xleftmargin=0.5cm,
    framexleftmargin=0.5cm,
    showstringspaces=false,
    tabsize=2
}

% Estilos de títulos (negro)
\titleformat{\section}{\Large\bfseries}{\thesection}{1em}{}
\titleformat{\subsection}{\large\bfseries}{\thesubsection}{1em}{}
\titleformat{\subsubsection}{\normalsize\bfseries}{\thesubsubsection}{1em}{}

% Configuración de encabezados y pies de página
\pagestyle{fancy}
\fancyhf{}
\fancyhead[L]{\textbf{AKEMY}}
\fancyhead[R]{\textcolor{gray}{Manual Técnico}}
\fancyfoot[C]{\thepage}
\renewcommand{\headrulewidth}{0.4pt}
\renewcommand{\footrulewidth}{0.4pt}

\begin{document}

% =====================================================
% PORTADA
% =====================================================
\begin{titlepage}
    \centering
    \vspace*{0.5cm}
    
    % Universidad
    {\Large\bfseries UNIVERSIDAD NACIONAL DEL ALTIPLANO}\\[0.3cm]
    {\large Facultad de Ingeniería Estadística e Informática}\\[0.2cm]
    {\large Escuela Profesional de Ingeniería Estadística e Informática}\\[1.5cm]
    
    \rule{\textwidth}{0.5pt}\\[1cm]
    
    % Título
    {\Huge\bfseries MANUAL TÉCNICO}\\[0.5cm]
    {\LARGE Sistema E-commerce AKEMY}\\[0.3cm]
    {\Large Librería y Papelería Online}\\[1cm]
    
    \rule{\textwidth}{0.5pt}\\[1.5cm]
    
    % Información del curso
    \begin{tabular}{ll}
        \textbf{Curso:} & Ingeniería de Software II \\
        \textbf{Docente:} & Laura Murillo Ramiro Pedro \\
    \end{tabular}\\[1.5cm]
    
    % Información del estudiante
    \begin{tabular}{ll}
        \textbf{Estudiante:} & Junior Beltran Huaraya Chipana \\
        \textbf{GitHub:} & \url{https://github.com/BeltranHC} \\
    \end{tabular}\\[1.5cm]
    
    % Información del proyecto
    \begin{tabular}{ll}
        \textbf{Proyecto:} & AKEMY - Sistema E-commerce \\
        \textbf{Tipo:} & Aplicación Web Full Stack \\
        \textbf{URL Producción:} & \url{https://akemy.app} \\
    \end{tabular}
    
    \vfill
    
    {\large\bfseries Puno - Perú}\\[0.3cm]
    {\large Diciembre 2025}
    
\end{titlepage}

% =====================================================
% TABLA DE CONTENIDOS
% =====================================================
\tableofcontents
\newpage

% =====================================================
% 1. INTRODUCCIÓN TÉCNICA
% =====================================================
\section{Introducción Técnica}

\subsection{Propósito del Manual Técnico}

El presente manual técnico tiene como objetivo documentar de manera exhaustiva la arquitectura, diseño e implementación del sistema de comercio electrónico AKEMY. Este documento está destinado a proporcionar toda la información técnica necesaria para comprender, mantener, extender y desplegar el sistema.

El manual cubre los siguientes aspectos:
\begin{itemize}
    \item Arquitectura general del sistema y sus componentes
    \item Tecnologías y frameworks utilizados en el desarrollo
    \item Diseño de la base de datos y API REST
    \item Medidas de seguridad implementadas
    \item Procedimientos de configuración y despliegue
    \item Integración con servicios externos (Mercado Pago, Resend)
    \item Consideraciones de escalabilidad y rendimiento
\end{itemize}

\subsection{Alcance del Documento}

Este manual abarca la totalidad del sistema AKEMY, incluyendo:

\begin{itemize}
    \item \textbf{Backend API:} Desarrollado con NestJS 10.3 y PostgreSQL
    \item \textbf{Frontend Web:} Desarrollado con Next.js 15 y React 18
    \item \textbf{Base de Datos:} PostgreSQL 16 alojada en Neon (Serverless)
    \item \textbf{Servicios Externos:} Mercado Pago (pagos), Resend (emails)
    \item \textbf{Infraestructura:} Vercel (frontend), Render (backend)
\end{itemize}

\subsection{Público Objetivo}

Este documento está dirigido a:
\begin{itemize}
    \item Desarrolladores que trabajarán en el mantenimiento o extensión del sistema
    \item Arquitectos de software que necesiten comprender la estructura del sistema
    \item Administradores de sistemas encargados del despliegue y operación
    \item Auditores técnicos que evalúen la seguridad y calidad del código
\end{itemize}

% =====================================================
% 2. DESCRIPCIÓN GENERAL DEL SISTEMA
% =====================================================
\section{Descripción General del Sistema}

\subsection{Visión Técnica del Sistema}

AKEMY es una plataforma de comercio electrónico diseñada específicamente para el sector de papelería y librería en el mercado peruano. El sistema implementa una arquitectura moderna de tres capas con separación clara entre la presentación (Frontend), lógica de negocio (Backend API) y persistencia de datos (PostgreSQL).

La plataforma está diseñada para ser:
\begin{itemize}
    \item \textbf{Escalable:} Capaz de manejar crecimiento en usuarios y transacciones
    \item \textbf{Segura:} Implementando múltiples capas de protección
    \item \textbf{Responsiva:} Funcionando óptimamente en dispositivos móviles y desktop
    \item \textbf{Mantenible:} Con código modular y bien documentado
\end{itemize}

\subsection{Problema Abordado desde el Enfoque Tecnológico}

El sistema aborda los siguientes desafíos técnicos:

\begin{enumerate}
    \item \textbf{Gestión de Inventario en Tiempo Real:} Control preciso de stock con actualizaciones inmediatas al procesar pedidos.
    
    \item \textbf{Procesamiento de Pagos Seguro:} Integración con Mercado Pago para pagos con tarjeta, cumpliendo estándares PCI DSS.
    
    \item \textbf{Comunicación en Tiempo Real:} Sistema de chat cliente-soporte mediante WebSockets (Socket.io).
    
    \item \textbf{Autenticación Robusta:} Sistema de doble token (Access + Refresh) con JWT.
    
    \item \textbf{Prevención de Ataques:} Sanitización XSS, rate limiting, headers de seguridad HTTP.
\end{enumerate}

\subsection{Funcionalidades Técnicas Principales}

\begin{table}[H]
\centering
\begin{tabular}{ll}
\toprule
\textbf{Módulo} & \textbf{Funcionalidad} \\
\midrule
Auth & Registro, login, JWT, refresh tokens, verificación email \\
Products & CRUD productos, variantes, imágenes, stock, filtros avanzados \\
Cart & Carrito persistente, cálculo con ofertas \\
Orders & Creación pedidos, flujo de estados, historial \\
Payments & Integración Mercado Pago, webhooks \\
Chat & WebSocket bidireccional, tiempo real \\
Dashboard & Gráficos de ventas, trends, carritos abandonados \\
Upload & Cloudinary para imágenes de productos y banners \\
Offers & Promociones con fechas, descuentos automáticos \\
Coupons & Validación, límites de uso, descuentos \\
Reviews & Reseñas con moderación, calificaciones \\
\bottomrule
\end{tabular}
\caption{Módulos técnicos principales}
\end{table}

\subsection{Flujo General de Operación del Sistema}

El flujo típico de una transacción en el sistema es:

\begin{enumerate}
    \item El cliente navega el catálogo (Frontend consulta API de productos)
    \item Agrega productos al carrito (API calcula totales con ofertas activas)
    \item Procede al checkout (validación de datos, stock, cupones)
    \item Selecciona método de pago:
    \begin{itemize}
        \item \textbf{Mercado Pago:} Redirección a checkout externo, webhook de confirmación
        \item \textbf{Contra entrega:} Pedido creado con estado PENDING
    \end{itemize}
    \item Sistema actualiza stock, envía emails, asigna puntos de fidelidad
    \item Administrador gestiona el pedido hasta su entrega
\end{enumerate}

% =====================================================
% 3. ARQUITECTURA DEL SISTEMA
% =====================================================
\section{Arquitectura del Sistema}

\subsection{Arquitectura General del Sistema}

El sistema utiliza una arquitectura de microservicios monolítica modular, donde el backend está organizado en módulos independientes pero desplegados como una única unidad.

\begin{lstlisting}[caption=Diagrama de Arquitectura General]
+----------------------------------------------------------+
|                    CLIENTE (Browser)                      |
|  +----------------------------------------------------+  |
|  |              Frontend (Next.js 15)                  |  |
|  |   React 18 | TailwindCSS | Zustand | TanStack Query |  |
|  +----------------------------------------------------+  |
+----------------------------------------------------------+
                            |
            HTTP/HTTPS (REST API) + WebSocket (Socket.io)
                            v
+----------------------------------------------------------+
|                   BACKEND (NestJS 10.3)                   |
|  +----------------------------------------------------+  |
|  |  Modules: Auth, Products, Cart, Orders, Payments,   |  |
|  |  Chat, Offers, Coupons, Reviews, Returns, Loyalty   |  |
|  +----------------------------------------------------+  |
|  | JWT | Rate Limiting | CORS | Helmet | Sanitize XSS  |  |
|  +----------------------------------------------------+  |
|                           |                              |
|                      Prisma ORM                          |
|                           v                              |
|  +----------------------------------------------------+  |
|  |           PostgreSQL 16 (Neon Serverless)          |  |
|  +----------------------------------------------------+  |
+----------------------------------------------------------+
                            |
        +-------------------+-------------------+
        v                                       v
+------------------+                   +------------------+
|   Mercado Pago   |                   |      Resend      |
|   (Pagos)        |                   |   (Emails)       |
+------------------+                   +------------------+
\end{lstlisting}

\subsubsection{Componentes de la Arquitectura}

\begin{itemize}
    \item \textbf{Frontend (Vercel):} Aplicación Next.js 15 con App Router, renderizado del lado del servidor (SSR) y generación estática (SSG).
    
    \item \textbf{Backend API (Render):} API REST con NestJS 10.3, autenticación JWT, WebSockets para chat en tiempo real.
    
    \item \textbf{Base de Datos (Neon):} PostgreSQL 16 serverless con connection pooling mediante PgBouncer.
    
    \item \textbf{Servicios Externos:}
    \begin{itemize}
        \item Mercado Pago: Procesamiento de pagos con tarjeta
        \item Resend: Envío de emails transaccionales
    \end{itemize}
\end{itemize}

\subsubsection{Flujo de Comunicación entre Componentes}

\begin{table}[H]
\centering
\begin{tabular}{llll}
\toprule
\textbf{Origen} & \textbf{Destino} & \textbf{Protocolo} & \textbf{Propósito} \\
\midrule
Frontend & Backend & HTTPS (REST) & Operaciones CRUD \\
Frontend & Backend & WSS (Socket.io) & Chat tiempo real \\
Backend & Neon DB & PostgreSQL & Persistencia \\
Backend & Mercado Pago & HTTPS & Crear preferencias \\
Mercado Pago & Backend & HTTPS (Webhook) & Confirmar pagos \\
Backend & Resend & HTTPS & Enviar emails \\
\bottomrule
\end{tabular}
\caption{Flujo de comunicación entre componentes}
\end{table}

\subsubsection{Principios Arquitectónicos Aplicados}

\begin{itemize}
    \item \textbf{Separación de Responsabilidades (SoC):} Cada módulo tiene una responsabilidad única y bien definida.
    
    \item \textbf{Inyección de Dependencias:} NestJS proporciona un contenedor IoC nativo para gestionar dependencias.
    
    \item \textbf{Repository Pattern:} Prisma ORM abstrae el acceso a datos con un cliente tipado.
    
    \item \textbf{DTO Pattern:} Objetos de transferencia de datos para validación y serialización.
    
    \item \textbf{Guard Pattern:} Protección de rutas mediante guards de autenticación y roles.
\end{itemize}

\subsection{Arquitectura del Backend}

\subsubsection{Organización Interna del Backend}

\begin{lstlisting}[caption=Estructura del Backend]
backend/src/
|-- main.ts                 # Punto de entrada
|-- app.module.ts           # Modulo raiz
|-- prisma/                 # Servicio Prisma
|-- common/                 # Pipes, Guards, Decoradores
|   |-- pipes/
|   |   +-- sanitize-input.pipe.ts
|   +-- decorators/
|-- auth/                   # Autenticacion JWT
|-- users/                  # Gestion usuarios
|-- products/               # Catalogo
|-- categories/             # Categorias jerarquicas
|-- brands/                 # Marcas
|-- cart/                   # Carrito de compras
|-- orders/                 # Pedidos
|-- payments/               # Mercado Pago
|-- chat/                   # WebSocket Gateway
|-- offers/                 # Promociones
|-- coupons/                # Cupones
|-- reviews/                # Resenas
|-- returns/                # Devoluciones
|-- loyalty/                # Puntos fidelidad
|-- wishlist/               # Lista deseos
|-- comparison/             # Comparador
|-- banners/                # Banners
|-- settings/               # Configuracion
|-- dashboard/              # Estadisticas
|-- upload/                 # Archivos
+-- mail/                   # Emails (Resend)
\end{lstlisting}

\subsubsection{Capa de Rutas y Controladores}

Los controladores definen los endpoints REST y están decorados con metadatos de Swagger para documentación automática:

\begin{lstlisting}[language=Java, caption=Ejemplo de Controlador]
@Controller('products')
@ApiTags('Products')
export class ProductsController {
  @Get()
  @Public()
  @ApiOperation({ summary: 'Listar productos' })
  findAll(@Query() filters: FilterProductsDto) {
    return this.productsService.findAll(filters);
  }
  
  @Post()
  @UseGuards(JwtAuthGuard, RolesGuard)
  @Roles('ADMIN', 'SUPERADMIN')
  create(@Body() dto: CreateProductDto) {
    return this.productsService.create(dto);
  }
}
\end{lstlisting}

\subsubsection{Capa de Servicios y Lógica de Negocio}

La lógica de negocio está encapsulada en servicios inyectables:

\begin{lstlisting}[language=Java, caption=Ejemplo de Servicio]
@Injectable()
export class CartService {
  constructor(
    private prisma: PrismaService,
    private offersService: OffersService,
  ) {}

  async addItem(dto: AddToCartDto, userId?: string) {
    // Validar stock
    // Verificar ofertas activas
    // Calcular totales
    // Persistir cambios
  }
}
\end{lstlisting}

\subsubsection{Persistencia de Datos y Base de Datos}

El sistema utiliza Prisma ORM para la persistencia de datos:

\begin{itemize}
    \item \textbf{Schema declarativo:} Modelo de datos definido en \texttt{schema.prisma}
    \item \textbf{Migraciones:} Control de versiones del esquema de base de datos
    \item \textbf{Cliente tipado:} Consultas con autocompletado y type-safety
    \item \textbf{Connection Pooling:} PgBouncer integrado en Neon
\end{itemize}

\subsubsection{Seguridad, Autenticación y Control de Acceso}

\begin{itemize}
    \item \textbf{JWT:} Tokens firmados con secreto configurable
    \item \textbf{Refresh Tokens:} Renovación automática de sesiones
    \item \textbf{Guards:} JwtAuthGuard, RolesGuard para protección de rutas
    \item \textbf{Rate Limiting:} Throttler con límites por endpoint
    \item \textbf{Sanitización:} Pipe global que limpia inputs de XSS
\end{itemize}

\subsubsection{Comunicación en Tiempo Real}

El módulo de chat utiliza WebSockets mediante Socket.io:

\begin{lstlisting}[language=Java, caption=WebSocket Gateway]
@WebSocketGateway({
  cors: { origin: FRONTEND_URL, credentials: true },
  namespace: '/chat',
  transports: ['polling', 'websocket'],
})
export class ChatGateway {
  @SubscribeMessage('sendMessage')
  handleMessage(client: Socket, payload: SendMessageDto) {
    // Persistir mensaje
    // Emitir a sala
    // Actualizar contador no leidos
  }
}
\end{lstlisting}

\subsection{Arquitectura del Frontend}

\subsubsection{Estructura del Proyecto Frontend}

\begin{lstlisting}[caption=Estructura del Frontend]
frontend/src/
|-- app/                    # App Router (Next.js 15)
|   |-- (shop)/             # Paginas de tienda
|   |-- admin/              # Panel administracion
|   |-- checkout/           # Checkout y pagos
|   |   |-- success/
|   |   |-- failure/
|   |   +-- pending/
|   +-- cuenta/             # Area cliente
|-- components/
|   |-- layout/             # Header, Footer, etc.
|   |-- ui/                 # shadcn/ui components
|   |-- products/           # ProductCard, etc.
|   |-- cart/               # CartDrawer
|   +-- chat/               # ChatWidget
|-- lib/
|   |-- api.ts              # Cliente API (Axios)
|   |-- store.ts            # Estado global (Zustand)
|   |-- socket.tsx          # Provider WebSocket
|   +-- utils.ts            # Funciones helper
\end{lstlisting}

\subsubsection{Capa de Presentación y Componentes}

\begin{itemize}
    \item \textbf{React 18:} Componentes funcionales con hooks
    \item \textbf{TailwindCSS 3.4:} Estilos utilitarios con diseño responsive
    \item \textbf{shadcn/ui:} Componentes accesibles basados en Radix UI
    \item \textbf{Lucide React:} Iconografía consistente
\end{itemize}

\subsubsection{Gestión del Estado y Acceso a Datos}

\begin{itemize}
    \item \textbf{Zustand:} Estado global para carrito, autenticación y wishlist
    \item \textbf{TanStack Query:} Data fetching, caching y sincronización
    \item \textbf{React Hook Form:} Manejo de formularios con validación Zod
\end{itemize}

\subsubsection{Comunicación en Tiempo Real}

El frontend se conecta al WebSocket del backend mediante un provider Context:

\begin{lstlisting}[language=Java, caption=Provider Socket.io]
export function SocketProvider({ children }) {
  const socket = useMemo(() => 
    io(SOCKET_URL, {
      withCredentials: true,
      transports: ['polling', 'websocket'],
    }), []
  );
  
  return (
    <SocketContext.Provider value={socket}>
      {children}
    </SocketContext.Provider>
  );
}
\end{lstlisting}

\subsection{Comunicación e Integración entre Servicios}

\subsubsection{Comunicación Frontend–Backend}

Todas las llamadas HTTP utilizan Axios con interceptores configurados:

\begin{lstlisting}[language=Java, caption=Cliente API]
const api = axios.create({
  baseURL: API_URL,
  withCredentials: true,
});

api.interceptors.request.use((config) => {
  const token = getAccessToken();
  if (token) {
    config.headers.Authorization = `Bearer ${token}`;
  }
  return config;
});
\end{lstlisting}

\subsubsection{Comunicación Backend–Mercado Pago}

La integración con Mercado Pago sigue el flujo de preferencias:

\begin{enumerate}
    \item Backend crea preferencia con items del pedido
    \item Frontend redirige a checkout de Mercado Pago
    \item Usuario completa pago en plataforma segura
    \item Mercado Pago envía webhook con resultado
    \item Backend actualiza estado del pedido
\end{enumerate}

\subsubsection{Consideraciones de Seguridad en la Integración}

\begin{itemize}
    \item \textbf{HTTPS obligatorio:} Todas las comunicaciones son cifradas
    \item \textbf{Webhooks verificados:} Validación de origen de Mercado Pago
    \item \textbf{Tokens seguros:} Access tokens con expiración corta (15min)
    \item \textbf{CORS estricto:} Solo orígenes permitidos pueden acceder
\end{itemize}

% =====================================================
% 4. TECNOLOGÍAS UTILIZADAS
% =====================================================
\section{Tecnologías Utilizadas}

\subsection{Tecnologías del Backend}

\begin{table}[H]
\centering
\begin{tabular}{lll}
\toprule
\textbf{Tecnología} & \textbf{Versión} & \textbf{Propósito} \\
\midrule
NestJS & 10.3 & Framework principal (Node.js) \\
TypeScript & 5.3 & Lenguaje de programación \\
Prisma & 5.8 & ORM y migraciones \\
PostgreSQL & 16 & Base de datos relacional \\
Socket.io & 4.8 & WebSockets tiempo real \\
Mercado Pago SDK & 2.11 & Pasarela de pagos \\
Resend & 6.6 & Emails transaccionales \\
JWT (@nestjs/jwt) & 10.2 & Autenticación \\
Bcrypt & 2.4 & Hashing contraseñas \\
Cloudinary & 2.5 & Almacenamiento de imágenes \\
Helmet & 7.1 & Headers seguridad HTTP \\
Sanitize-html & 2.17 & Prevención XSS \\
Swagger & 7.2 & Documentación API \\
Class-validator & 0.14 & Validación DTOs \\
\bottomrule
\end{tabular}
\caption{Stack tecnológico del Backend}
\end{table}

\subsection{Tecnologías del Frontend}

\begin{table}[H]
\centering
\begin{tabular}{lll}
\toprule
\textbf{Tecnología} & \textbf{Versión} & \textbf{Propósito} \\
\midrule
Next.js & 15 & Framework React (App Router) \\
React & 18 & Biblioteca UI \\
TypeScript & 5.3 & Lenguaje de programación \\
TailwindCSS & 3.4 & Framework CSS utilitario \\
shadcn/ui & Latest & Componentes accesibles \\
Radix UI & Latest & Primitivos UI \\
Zustand & 5 & Gestión estado global \\
TanStack Query & 5 & Data fetching/caching \\
React Hook Form & 7.53 & Manejo formularios \\
Zod & 3.22 & Validación esquemas \\
Recharts & 2.15 & Gráficos para dashboard \\
Socket.io-client & 4.8 & Cliente WebSocket \\
Lucide React & Latest & Iconografía \\
\bottomrule
\end{tabular}
\caption{Stack tecnológico del Frontend}
\end{table}

\subsection{Tecnologías de Infraestructura}

\begin{table}[H]
\centering
\begin{tabular}{lll}
\toprule
\textbf{Servicio} & \textbf{Plataforma} & \textbf{Propósito} \\
\midrule
Frontend Hosting & Vercel & Deploy Next.js con SSR \\
Backend Hosting & Render & Deploy NestJS \\
Base de Datos & Neon & PostgreSQL Serverless \\
Imágenes & Cloudinary & CDN y almacenamiento \\
Pagos & Mercado Pago & Procesamiento tarjetas \\
Emails & Resend & Transaccionales \\
Control de Versiones & GitHub & Repositorio código \\
\bottomrule
\end{tabular}
\caption{Infraestructura de producción}
\end{table}

% =====================================================
% 5. DISEÑO DEL SISTEMA
% =====================================================
\section{Diseño del Sistema}

\subsection{Diseño de Componentes del Sistema}

El sistema está compuesto por los siguientes componentes principales:

\begin{enumerate}
    \item \textbf{Módulo de Autenticación:} Gestión de identidad, tokens JWT, verificación email.
    
    \item \textbf{Módulo de Catálogo:} Productos, categorías, marcas, variantes, imágenes.
    
    \item \textbf{Módulo de Comercio:} Carrito, pedidos, ofertas, cupones.
    
    \item \textbf{Módulo de Pagos:} Integración Mercado Pago, webhooks.
    
    \item \textbf{Módulo de Comunicación:} Chat en tiempo real, notificaciones.
    
    \item \textbf{Módulo de Fidelización:} Puntos, wishlist, reseñas.
\end{enumerate}

\subsection{Diseño de la Base de Datos}

Las entidades principales del sistema son:

\begin{table}[H]
\centering
\begin{tabular}{ll}
\toprule
\textbf{Entidad} & \textbf{Descripción} \\
\midrule
User & Usuarios (clientes y administradores) \\
Product & Catálogo de productos \\
ProductImage & Imágenes de productos \\
ProductVariant & Variantes (color, tamaño) \\
Category & Categorías jerárquicas \\
Brand & Marcas de productos \\
Cart / CartItem & Carrito de compras \\
Order / OrderItem & Pedidos y líneas de pedido \\
Coupon & Cupones de descuento \\
Offer / OfferProduct & Promociones \\
Review & Reseñas de productos \\
Conversation / Message & Chat \\
\bottomrule
\end{tabular}
\caption{Entidades principales de la base de datos}
\end{table}

\subsection{Diseño de la API REST}

La API sigue principios RESTful con las siguientes convenciones:

\begin{itemize}
    \item \textbf{Prefijo global:} \texttt{/api}
    \item \textbf{Versionado:} Implícito en URL base
    \item \textbf{Formato:} JSON para request/response
    \item \textbf{Autenticación:} Bearer token en header Authorization
    \item \textbf{Códigos HTTP:} 200 OK, 201 Created, 400 Bad Request, 401 Unauthorized, 403 Forbidden, 404 Not Found
\end{itemize}

\textbf{Documentación Swagger:} Disponible en \url{https://akemy-backend.onrender.com/api/docs}

\subsection{Diseño de la Comunicación en Tiempo Real}

\begin{table}[H]
\centering
\begin{tabular}{lll}
\toprule
\textbf{Evento} & \textbf{Dirección} & \textbf{Payload} \\
\midrule
startConversation & Cliente → Servidor & \{ subject?: string \} \\
sendMessage & Cliente → Servidor & \{ conversationId, content \} \\
joinConversation & Cliente → Servidor & \{ conversationId \} \\
markAsRead & Cliente → Servidor & \{ conversationId \} \\
newMessage & Servidor → Cliente & Message object \\
conversation & Servidor → Cliente & Conversation object \\
unreadCount & Servidor → Cliente & \{ count: number \} \\
\bottomrule
\end{tabular}
\caption{Eventos WebSocket del sistema de chat}
\end{table}

% =====================================================
% 6. SEGURIDAD DEL SISTEMA
% =====================================================
\section{Seguridad del Sistema}

\subsection{Autenticación y Autorización}

El sistema implementa un esquema de doble token JWT:

\begin{itemize}
    \item \textbf{Access Token (15 minutos):} Token de corta duración para acceso a recursos.
    \item \textbf{Refresh Token (7 días):} Token de larga duración almacenado hasheado en BD.
\end{itemize}

\begin{lstlisting}[caption=Estructura del JWT]
{
  "sub": "user-uuid",
  "email": "usuario@email.com",
  "role": "CUSTOMER",
  "firstName": "Nombre",
  "iat": 1702656000,
  "exp": 1702656900
}
\end{lstlisting}

\subsection{Gestión de Roles y Permisos}

\begin{table}[H]
\centering
\begin{tabular}{ll}
\toprule
\textbf{Rol} & \textbf{Permisos} \\
\midrule
CUSTOMER & Compras, perfil, historial, chat \\
ADMIN & Lo anterior + gestión productos, pedidos \\
SUPERADMIN & Acceso total, configuración sistema \\
EDITOR & Gestión de contenido (banners, categorías) \\
PRODUCT\_MANAGER & Gestión de productos e inventario \\
\bottomrule
\end{tabular}
\caption{Roles y permisos del sistema}
\end{table}

\subsection{Protección de Datos y Privacidad}

\begin{table}[H]
\centering
\begin{tabular}{lll}
\toprule
\textbf{Medida} & \textbf{Implementación} & \textbf{Descripción} \\
\midrule
Hashing contraseñas & bcryptjs & Salt rounds: 10 \\
Sanitización XSS & sanitize-html & Pipe global \\
Rate Limiting & @nestjs/throttler & Límites por endpoint \\
Headers HTTP & Helmet & CSP, HSTS, X-XSS-Protection \\
Validación inputs & class-validator & DTOs estrictos \\
CORS & @nestjs/cors & Orígenes permitidos \\
\bottomrule
\end{tabular}
\caption{Medidas de protección de datos}
\end{table}

\subsection{Rate Limiting por Endpoint}

\begin{table}[H]
\centering
\begin{tabular}{lcc}
\toprule
\textbf{Endpoint} & \textbf{Límite} & \textbf{Período} \\
\midrule
/auth/login & 5 intentos & 60 segundos \\
/auth/admin/login & 3 intentos & 60 segundos \\
General & 100 requests & 60 segundos \\
\bottomrule
\end{tabular}
\caption{Configuración de Rate Limiting}
\end{table}

\subsection{Seguridad en la Comunicación}

\begin{itemize}
    \item \textbf{HTTPS obligatorio:} TLS en todas las comunicaciones
    \item \textbf{HSTS:} HTTP Strict Transport Security habilitado
    \item \textbf{CSP:} Content Security Policy para prevenir inyecciones
    \item \textbf{Cookies seguras:} httpOnly, secure, sameSite
\end{itemize}

% =====================================================
% 7. CONFIGURACIÓN Y DESPLIEGUE
% =====================================================
\section{Configuración y Despliegue}

\subsection{Requisitos Técnicos del Sistema}

\begin{table}[H]
\centering
\begin{tabular}{ll}
\toprule
\textbf{Requisito} & \textbf{Especificación} \\
\midrule
Node.js & 20.x o superior \\
PostgreSQL & 16.x (o Neon Serverless) \\
npm & 10.x o superior \\
Memoria RAM & Mínimo 512MB por servicio \\
\bottomrule
\end{tabular}
\caption{Requisitos técnicos}
\end{table}

\subsection{Variables de Entorno del Backend}

\begin{lstlisting}[caption=Variables de entorno (.env)]
# Servidor
NODE_ENV=production
PORT=3001

# Base de Datos (Neon con pooling)
DATABASE_URL="postgresql://...@...-pooler...?sslmode=require&pgbouncer=true"

# JWT
JWT_SECRET=clave-secreta-muy-larga-y-segura
JWT_REFRESH_SECRET=otra-clave-secreta-diferente
JWT_EXPIRATION=15m
JWT_REFRESH_EXPIRATION=7d

# CORS
FRONTEND_URL=https://akemy.app

# Mercado Pago
MERCADO_PAGO_ACCESS_TOKEN=APP_USR-xxx
MERCADO_PAGO_PUBLIC_KEY=APP_USR-xxx
BACKEND_URL=https://akemy-backend.onrender.com

# Resend (Emails)
RESEND_API_KEY=re_xxx
EMAIL_FROM="AKEMY <noreply@akemy.app>"

# Rate Limiting
THROTTLE_TTL=60
THROTTLE_LIMIT=100
\end{lstlisting}

\subsection{Despliegue Mediante Docker Compose}

\begin{lstlisting}[language=bash, caption=Comandos Docker]
# Construir e iniciar servicios
docker-compose up -d

# Ver logs en tiempo real
docker-compose logs -f

# Ejecutar migraciones
docker-compose exec backend npx prisma migrate deploy

# Ejecutar seed de datos iniciales
docker-compose exec backend npx prisma db seed

# Detener servicios
docker-compose down
\end{lstlisting}

\subsection{Despliegue en Producción}

\subsubsection{Frontend (Vercel)}

\begin{enumerate}
    \item Conectar repositorio GitHub a Vercel
    \item Configurar variable \texttt{NEXT\_PUBLIC\_API\_URL}
    \item Deploy automático en cada push a main
\end{enumerate}

\subsubsection{Backend (Render)}

\begin{enumerate}
    \item Crear Web Service conectado a GitHub
    \item Configurar Build Command: \texttt{npm install \&\& npm run build}
    \item Configurar Start Command: \texttt{npm run start:prod}
    \item Agregar todas las variables de entorno
    \item Deploy automático en cada push
\end{enumerate}

% =====================================================
% 8. INTEGRACIÓN DE PAGOS (MERCADO PAGO)
% =====================================================
\section{Integración de Pagos con Mercado Pago}

\subsection{Rol del Servicio de Pagos}

El módulo de pagos gestiona la integración con Mercado Pago para procesar pagos con tarjeta de crédito y débito. El flujo está diseñado para que los datos sensibles de tarjetas nunca pasen por nuestros servidores.

\subsection{Flujo de Pago}

\begin{enumerate}
    \item \textbf{Crear Preferencia:} Backend crea preferencia con items del pedido
    \item \textbf{Redirección:} Frontend redirige al checkout de Mercado Pago
    \item \textbf{Pago:} Usuario ingresa datos en plataforma segura de MP
    \item \textbf{Webhook:} MP notifica resultado al backend
    \item \textbf{Actualización:} Backend actualiza estado del pedido
    \item \textbf{Redirección:} Usuario vuelve a página de éxito/fallo/pendiente
\end{enumerate}

\subsection{Endpoints del Módulo de Pagos}

\begin{table}[H]
\centering
\begin{tabular}{lll}
\toprule
\textbf{Método} & \textbf{Endpoint} & \textbf{Descripción} \\
\midrule
POST & /payments/create-preference & Crear preferencia de pago \\
POST & /payments/webhook & Recibir notificación de MP \\
GET & /payments/status/:id & Consultar estado del pago \\
GET & /payments/config & Obtener public key \\
\bottomrule
\end{tabular}
\caption{Endpoints del módulo de pagos}
\end{table}

\subsection{Tarjetas de Prueba}

\begin{table}[H]
\centering
\begin{tabular}{llll}
\toprule
\textbf{Tipo} & \textbf{Número} & \textbf{CVV} & \textbf{Estado} \\
\midrule
Visa & 4509 9535 6623 3704 & 123 & Aprobada \\
Mastercard & 5031 7557 3453 0604 & 123 & Aprobada \\
\bottomrule
\end{tabular}
\caption{Tarjetas de prueba de Mercado Pago}
\end{table}

% =====================================================
% 9. CONSIDERACIONES DE ESCALABILIDAD
% =====================================================
\section{Consideraciones de Escalabilidad y Rendimiento}

\subsection{Connection Pooling (Neon)}

La base de datos utiliza PgBouncer para connection pooling, evitando el overhead de crear conexiones nuevas en cada request. Configuración mediante parámetro \texttt{\&pgbouncer=true} en DATABASE\_URL.

\subsection{Manejo de Concurrencia}

\begin{itemize}
    \item \textbf{WebSockets:} Socket.io maneja múltiples conexiones concurrentes
    \item \textbf{Rate Limiting:} Previene sobrecarga por abuso
    \item \textbf{Transacciones Prisma:} Operaciones atómicas en BD
\end{itemize}

\subsection{Caché y Optimización}

\begin{itemize}
    \item \textbf{TanStack Query:} Cache de datos en frontend (staleTime configurable)
    \item \textbf{Zustand:} Estado persistente del carrito
    \item \textbf{Next.js:} SSG para páginas estáticas, ISR para contenido dinámico
\end{itemize}

\subsection{Posibles Mejoras Futuras}

\begin{itemize}
    \item Implementar Redis para caché del backend
    \item CDN para assets estáticos
    \item Upgrade a Render Starter para eliminar cold starts
    \item Implementar queues para tareas pesadas (emails, reportes)
\end{itemize}

% =====================================================
% 10. MANTENIMIENTO Y EXTENSIBILIDAD
% =====================================================
\section{Mantenimiento y Extensibilidad del Sistema}

\subsection{Organización del Código Fuente}

El código sigue las convenciones de NestJS y Next.js:

\begin{itemize}
    \item Módulos independientes con responsabilidad única
    \item DTOs para validación de entrada
    \item Servicios para lógica de negocio
    \item Controllers para definición de endpoints
    \item Componentes React reutilizables
\end{itemize}

\subsection{Buenas Prácticas de Mantenimiento}

\begin{itemize}
    \item Mantener dependencias actualizadas (npm audit)
    \item Ejecutar migraciones en staging antes de producción
    \item Revisar logs de Render/Vercel periódicamente
    \item Monitorear métricas de Neon (conexiones, storage)
    \item Realizar backups periódicos de la base de datos
\end{itemize}

\subsection{Extensiones Futuras Sugeridas}

\begin{enumerate}
    \item Sistema de notificaciones push (FCM)
    \item Integración con más pasarelas de pago (Stripe, PayPal)
    \item App móvil nativa (React Native)
    \item Sistema de reportes y analytics
    \item Múltiples idiomas (i18n)
\end{enumerate}

% =====================================================
% 11. CONCLUSIONES TÉCNICAS
% =====================================================
\section{Conclusiones Técnicas}

El sistema AKEMY representa una implementación moderna de comercio electrónico con las siguientes características destacables:

\begin{itemize}
    \item \textbf{Arquitectura Modular:} Facilita el mantenimiento y extensión del sistema.
    
    \item \textbf{Stack Moderno:} NestJS + Next.js proporcionan una base sólida y escalable.
    
    \item \textbf{Seguridad Multicapa:} Múltiples medidas de protección implementadas.
    
    \item \textbf{Integración de Pagos:} Mercado Pago proporciona procesamiento seguro de tarjetas.
    
    \item \textbf{Comunicación en Tiempo Real:} Chat integrado para soporte al cliente.
    
    \item \textbf{Despliegue en la Nube:} Infraestructura serverless que escala automáticamente.
\end{itemize}

% =====================================================
% 12. LICENCIA DEL SOFTWARE
% =====================================================
\section{Licencia del Software}

Este proyecto está licenciado bajo la \textbf{Licencia MIT}, lo que permite:

\begin{itemize}
    \item Uso comercial y privado
    \item Modificación del código fuente
    \item Distribución del software
    \item Sublicenciamiento
\end{itemize}

La única condición es mantener el aviso de copyright original y la licencia en todas las copias.

% =====================================================
% 13. CRÉDITOS DEL PROYECTO
% =====================================================
\section{Créditos del Proyecto}

\begin{table}[H]
\centering
\begin{tabular}{ll}
\toprule
\textbf{Rol} & \textbf{Responsable} \\
\midrule
Desarrollo Full Stack & Junior Beltran Huaraya Chipana \\
Diseño UI/UX & Junior Beltran Huaraya Chipana \\
Arquitectura del Sistema & Junior Beltran Huaraya Chipana \\
\bottomrule
\end{tabular}
\caption{Equipo de desarrollo}
\end{table}

\textbf{Repositorio:} \url{https://github.com/BeltranHC/ecomerce_akemy}

\textbf{Contacto:} huaraya0804@email.com

% =====================================================
% FIN DEL DOCUMENTO
% =====================================================
\vspace{1cm}
\hrule
\vspace{0.5cm}
\centering
\textbf{AKEMY - Librería y Papelería Online}\\
\textit{Tu papelería favorita}\\[0.3cm]
\url{https://akemy.app}\\[0.5cm]
{\small Universidad Nacional del Altiplano - Puno, Perú}

\end{document}
